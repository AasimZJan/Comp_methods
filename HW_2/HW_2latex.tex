

\documentclass{article}
\usepackage[utf8]{inputenc}
\usepackage[english]{babel}
\usepackage[]{amsthm} %lets us use \begin{proof}
\usepackage[]{amssymb} %gives us the character \varnothing
\usepackage{graphicx}
\usepackage{hyperref}
\usepackage{multicol}


\title{Homework 1}
\author{Aasim Zahoor}
\date\today


\begin{document}
\maketitle 


\begin{center}
\section{Problems}
\end{center}
\textbf{Link}\vspace{1.5em}
\url{https://github.com/AasimZahoor/Comp_methods.git}
\vspace{1.5em}

\textbf{Problem 1}\vspace{1.5em}

In the code for Problem 1 I have defined five functions. They are:
\begin{itemize}
\item{\textbf{midpoint(f,a,b,n)}}\item{\textbf{trapezoid(f,a,b,n)}}\item{\textbf{simpson(f,a,b,n)}}\vspace{0.2em}

These functions have the same arguments and thet are f= 'the function being integrated', $[a,b]$='The range of integration', n='The number of steps'. All of them need to be inputted while calling the function.
\vspace{0.2em}

\item{\textbf{diff(f,a,h=0.000001)}}\vspace{0.2em}

This function differentiates the function f at point a. h represents the step size and has a default value of $0.000001$\vspace{0.2em}

\item{\textbf{g()}}\vspace{0.2em}

This is a test function and can be used to test these function. You would need to make changes in the code to test different function.
\vspace{0.2em}
\end{itemize}

\emph{\small{Testing the functions using sine function(the figure on the left), the results of the test(the figure on the right)}}
\begin{center}
\begin{multicols}{2}
	\begin{center}
        \includegraphics[scale=0.5]{Images/pb1test}
        \end{center}
\columnbreak
       \includegraphics[scale=0.37]{Images/pb1testp}
\end{multicols}
\end{center}
\vspace{0.2em}






\vspace{1.5em}
\textbf{Problem 2}\vspace{1.5em}



 
\vspace{1.5em}
\vspace{1.5em}
\textbf{Problem 3}\vspace{1.5em}

In this problem I have made a matrix class in which I have defined eight instances. \vspace{0.2em}

\begin{itemize}

\item{\textbf{init}}\vspace{0.2em} It has self and the matrix array as the argument. Here I defined 3 instance attributes, self.g gives back the array and helps me handle the array in a better way.
\item{\textbf{add(self,other)}}\vspace{0.2em} It takes self and the other(matrix) as the argument. I have included a check to see if addition is possible
\item{\textbf{mult(self,other)}}\vspace{0.2em} It takes self and the other(matrix) as the argument. I have included a check to see if multiplication is possible
\item{\textbf{tran(self)}}\vspace{0.2em}It takes self as argument and gives out Transpose as output.
\item{\textbf{trace(self)}}\vspace{0.2em}It takes self as argument and gives out Trace as output.
\item{\textbf{Det(self)}}\vspace{0.2em}It takes self as argument and gives out Determinant as output.

\begin{center}
\begin{multicols}{2}
	\begin{center}
        \includegraphics[scale=0.5]{Images/pb2test}
        \end{center}
\columnbreak
	\begin{center}
       \includegraphics[scale=0.37]{Images/pb2testp}
       \end{center}
\end{multicols}
\end{center}
\emph{\small{Testing the instances using a matrix(the figure on the left), the results of the test(the figure on the right)}}
\item{\textbf{LU(self)}}\vspace{0.2em}Takes in self as argument and gives out lower and upper triangular matrix (in that order). In the figure I multiplies the L and U matrix and got the original matrix back.

\begin{center}
	 \emph{\small{Testing the LU Instance using a matrix}}
	 
	 \vspace{0.2em}

        \includegraphics[scale=0.4]{Images/LUtest}
        
        
      
        
        \vspace{0.2em}
        
       \includegraphics[scale=0.4]{Images/LUP}
       
       \vspace{0.2em}
       
       \emph{\small{Results of the test}}
\end{center}

\item{\textbf{Inv(self)}}\vspace{0.2em}Takes self as the argument and gives out the inverse.

\begin{center}
	
	 
	

        \includegraphics[scale=0.4]{Images/invtest}
        
         \vspace{0.2em}
       \emph{\small{Testing the Inv Instance using a matrix}}
        
        \vspace{0.2em}
        
       \includegraphics[scale=0.3]{Images/invtestp}
       
       \vspace{0.2em}
       
       \emph{\small{Results of the test}}
       \end{center}

\vspace{0.2em}

\end{itemize}
\emph{\small{I multiplied the result of the inverse with the original and got the identity matrix}}
\begin{center}
        \includegraphics[scale=0.5]{Images/1}

       \includegraphics[scale=0.37]{Images/1p}
\end{center}

\vspace{0.2em}
\end{document}
